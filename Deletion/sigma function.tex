  the sigma function is defined as:

  $$
    \sigma_x(n)=\sum_{d\mid n} d^x
  $$

  when $x = 0$ is called the divisor function, that counts the number
  of positive divisors of n.


  Now, we are interested in find

  $$
    \sum_{d\mid n} \sigma_0(d)
  $$

  if $n$ is written as prime factorization:

  $$
    n = \prod_{i = 1}^{k} P_{i}^{e_k}
  $$

  we can demonstrate that:

  $$
    \sum_{d\mid n} \sigma_0(d) = \prod_{i = 1}^{k} g(e_k + 1)
  $$

  where $g(x)$ is the sum of the first x positive numbers:

  $$
    g(x) = (x * (x + 1)) / 2
  $$
